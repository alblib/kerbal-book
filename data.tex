\part{데이터 및 프로토콜}
\chapter{게임 데이터}
이 챕터의 내용은 주로 게임 내부 데이터 및 물리법칙에 대한 것이며, 
따라서 저작권은 게임제작사에 있다.%으며, 인용하고 있는 제3자의 저작권은 없는 것으로 해석한다. (독창성 결여)
%데이터는 게임 플레이 중 직접 확인할 수 있는 부분이며, 상이한 점을 발견하면 갱신 부탁드립니다.
데이터는 게임 플레이 중 직접 확인할 수 있는 부분이며, 상이한 점이 있다면 github에서 직접 갱신할 수 있다.

인용처는 다음과 같다.
\begin{itemize}
\item 게임에서 직접 확인한 정보 인용
\item 참고: ``Kerbal System Celestial Body Parameters in v0.18.2'' \url{https://docs.google.com/spreadsheets/d/1HwrFq6r2Wfzvghq8VYMNx0F2rG6nYMAEl_cyLj2EXjA/edit#gid=0}\footnote{I obtained this sheet from some forum but I forgot where it was. Sorry for that.}
\item 참고: \url{http://wiki.kerbalspaceprogram.com/}
\end{itemize}

\section{행성 데이터}
KSP에서 행성과 위성의 운동은 물리법칙의 시뮬레이션에 의한 것이 아니라 고정된 주기를 가진 고정된 운동궤도에 따른 것입니다. 공전주기와 공전궤도는 항성 또는 모행성이 아주 무겁다고 생각하고 만들어진 1체문제의 해이며, 모행성의 위성에 의한 흔들림 등은 전혀 고려되어 있지 않습니다.
\begin{landscape}
%\begin{center}
%\resizebox{\textheight}{!}{
\begin{adjustbox}{width=15.5 cm}%\textheight}
\begin{threeparttable}
\caption{(From Wiki and Forum)}
\begin{longtable}{|l|r|r|r|r|r|r|r|r|r|r|r|r|r|}
\hline
이름&$GM$\ttuna ($\mathrm{m}^3/\mathrm{s}^{2}$) &반지름&원일점$^\ast$&근일점$^\ast$%&공전주기\ttsecu
&자전주기$^\star$ &LAN$^\sharp$&INC$^\natural$&LPE$^\S$&MNA$^\flat$ &대기한계& 최고지점&지표면기압
\\\hline
Sol&$1.172\,332\,8\times 10^{18}$&261\,600\,000\,m&-&-&432\,000.000\,s&-&-&-&-&600\,000\,m&-&16\,kPa
\\\hline
Moho&$1.6860938\times 10^{11}$&250\,000\,m&6\,315\,765\,980\,m&4\,210\,510\,628\,m
%&2\,215\,754.220\,s
&1\,210\,000.000\,s&70.00$^\circ$&7.000\,$^\circ$&15.0\,$^\circ$&3.140\,rad&0&6\,817\,m&0
\\
Eve&$8.1717302\times 10^{12}$&700\,000\,m&9\,931\,011\,387\,m&9\,734\,357\,701\,m
%&5\,657\,995.1\,s
&80\,500.000\,s&15.00$^\circ$&2.100\,$^\circ$&0.0$^\circ$&3.140\,rad&90\,000\,m&7540\,m&506.625\,kPa
\\
Kerbin&$3.5316000\times 10^{12}$&600\,000\,m&13\,599\,840\,256\,m&13\,599\,840\,256\,m
&21\,549.425\,s&80.00$^\circ$&0&0.0$^\circ$&3.140\,rad&70\,000\,m&6764.1\,m&101.325\,kPa
\\
Duna&$3.0136321\times 10^{11}$&320\,000\,m&21\,783\,189\,162\,m&19\,669\,121\,365\,m
&65\,517.859\,s&135.50$^\circ$&0.060\,$^\circ$&0.0$^\circ$&3.140\,rad&50\,000\,m&7\,5**\,m&6.755\,kPa
\\
Dres&$2.1484489\times 10^{10}$&138\,000\,m&46\,761\,053\,522\,m&34\,917\,642\,884\,m
&34\,800.000\,s&280.00$^\circ$&5.000\,$^\circ$&90.0$^\circ$&3.140\,rad&0&5\,7**\,m&0
\\
Jool&$2.8252800\times 10^{14}$&6\,000\,000\,m&72\,212\,238\,387\,m&65\,334\,882\,252\,m
&36\,000.000\,s&52.00$^\circ$&1.304\,$^\circ$&0.0$^\circ$&0.100\,rad&200\,000\,m&-&1519.875\,kPa
\\
Eeloo&$7.4410815\times 10^{10}$&210\,000\,m&113\,549\,713\,200\,m&66\,687\,926\,800\,m
%&156\,992\,048.4\,s
&19\,460.000\,s&50.00$^\circ$&6.150\,$^\circ$&260.0$^\circ$&3.140\,rad&0&3\,9**\,m&0
\\\hline
\end{longtable}
\begin{tablenotes}
\item[$\dagger$]질량$\times$뉴턴중력상수. Standard gravitational parameter. KSP에서는 질량이 아니라 이 값이 exact하게 정의되어 있다.
%\item[$\ddagger$] 원일점과 근일점의 종속변수이나, 컴퓨터의 계산오차가 있으므로 게임 내부적으로 상수일 것으로 생각된다.
\item[$\star$] 1항성일. 절대계 기준으로, 1태양일과는 다르다.
\item[$\sharp$] Longitude of the Ascending Node
\item[$\natural$] Orbital Inclination
\item[$\S$] Argument of Periapsis
\item[$\flat$] Mean Anomaly at Epoch UT=0
\item[$\ast$] 태양 중심으로부터의 거리 (radius). Height(태양 표면으로부터의 거리)가 아님.
\end{tablenotes}
\end{threeparttable}
%}
\end{adjustbox}
%\end{center}
\end{landscape}

\iffalse

\paragraph{1 Solar Day}
Different than mean solar day, a solar day varies depending on the position on the solar orbit.
There are mainly two reasons of the variation of the solar day:
\begin{itemize}
\item eccentricity of the orbit, 
\item and axial tilt or obliquity, angle between equatorial plane and orbital plane.
\end{itemize}

Under these assumptions:
\begin{itemize}
\item the Sun is much more massive than a planet so that the orbit is perfectly ellipse,
\item and a day is much shorter than an year,
\end{itemize}
we can obtain the length of a solar day depending on the position.

First, let us obtain the angular velocity on certain position using Keplar's second law.
The total area enclosed by the elliptic orbit on the orbital plane is $\sqrt{r_p r_a}\frac{r_p+r_a}{2}\pi$ where $r_a$ is the apoapsis, and $r_p$ is the periapsis radius. Thus, the areal velocity is constantly $\sqrt{r_p r_a}\frac{r_p+r_a}{2}\frac{\pi}{T}$ where $T$ is the orbital period. In conclusion, the equation to obtain the relation between solar distance $r$ and angular velocity $\omega$ is:
\begin{align}
\frac{r^2 \omega}{2} = \sqrt{r_p r_a}\frac{r_p+r_a}{2}\frac{\pi}{T}
\\\nonumber\Rightarrow \omega = \sqrt{r_p r_a}\frac{r_p+r_a}{2r^2}\frac{2\pi}{T}.
\end{align}


\fi

%두가지 요인: 공전궤도의 이심률과 자전축과 공전평면과의 각도.
%넓이:sqrt{r_p r_a}*{r_p+r_a}/2*pi
%r cross v /2 = 넓이/주기
%omega= sqrt{r_p r_a}*{r_p+r_a}*pi/r^2/주기
%omega= sqrt{r_p r_a}*{r_p+r_a}/(2r^2)* 평균각속도 (공전평면상)

%-> 자전축을 수직으로 세우면?

%\section{위성 데이터}

%\begin{tabular}{|c|c|c|c|c|c|c|c|c|c|c|}
%\hline
%모행성&위성 이름& $GM$& 자전주기\footnote{1항성일}& 공전주기\footnote{1항성월}& 원지점&근지점&Sphere of influence&위성 반지름& 대기한계고도 & 지형 최대고도
%\\
%\multirow{2}{*}{Kerbin}& Mun
%\\&Minmus
%\end{tabular}
%\section{엔진 데이터}

%\begin{tabular}{|l|r|}
%\end{tabular}

%200 FuelUnit = 1t. (oxygen and liquid)
%\\10000 FuelUnit = 1t. (Xenon)
%density: expected around $1t/m^3$ (liquid)
%, 1.05949191932 $m^3$ = 1t.
%\\density:
%0.657420 $m^3$, 0.525t xenon. (xenon)
%0.0406->0.0371$m^3$,0.04t
%Globally, total fuel mass : tank mass = 8:1. (round bare tank. extra structure may weigh more.)
%Globally(xenon), total fuel mass : tank mass = 14:11. (round bare tank. extra structure may weigh more.)
\section{연료 종류별 성질}
\begin{adjustbox}{width=\textwidth}
\begin{threeparttable}
\caption{연료 종류별 성질}
\begin{tabular}[t]{|c|c|c|c|c|c|}
\hline
&Liquid/Oxydizer& Solid& RCS& Xenon& Ore
\\\hline
Density ($t/m^3$)\textsuperscript{*}&$\sim$1&
$\sim$1&0.5$\sim$4\textsuperscript{!}&$\sim$1&$\sim$ 1.3
\\\hline
Density (t/FuelUnit)& 1/200 & 3/400 & 1/250 & 1/10000 & 1/100
\\\hline
\makecell{Least Mass Ratio 
\vspace{-2pt}
\\
\vspace{2pt}
of $\dfrac{\text{(Tank Structure)}}{\text{(Fuel Capacity)}}$}
&1/8&0.231$\sim$0.2439\textsuperscript{\textdagger}&>0.1333$\sim$0.1775&11/14&1/6
\\\hline
Cost per FuelUnit&0.8(Liq)/0.18(Oxy)&0.6&1.2&4&0.02
\\\hline
\end{tabular}
    \begin{tablenotes}
    \item[!] Smaller tank can store more densely.
    \item[*] Measured from the graphic
    \item[\textdagger]  Including engine
    \end{tablenotes}
\end{threeparttable}
\end{adjustbox}
\chapter{프로토콜}
본 챕터에서는 우주선끼리의 충돌을 피하기 위해 우주영역을 구분하거나, 달에 착륙할 때 안전하고도 효율적인 표준 프로토콜을 정하는 등, 우주 비행에 있어서의 여러가지 프로토콜(약속)들을 저자가 개인적으로 정하여 제안하여 보는 바이다.
혹시 미래에 멀티플레이어 기능이 들어가게 된다면 유용하게 쓸 수 있을 것이다.
\section{Kerbin 궤도 프로토콜}
타원궤도의 주기는 다음과 같다. $a$는 장축의 반이다.
\begin{align}
T = 2\pi \sqrt{\frac{a^3}{GM}}
\end{align}
\begin{adjustbox}{width=\textwidth}
\begin{threeparttable}
\caption{궤도 고도 프로토콜}
\begin{tabular}{|r|r|l|}
\hline
궤도고도 & 궤도이름& 용도
\\\hline
80\,km$\pm$5\,km&Insertion Orbit& 지표면에서부터 궤도에 처음 오른 우주선이 
\\&&들어서는 궤도
\\100\,km$\pm$5\,km&&
\\120\,km$\pm$5\,km&Space Station Orbit& 우주정거장이 들어설 궤도. 궤도 조정을 할 수 있게 
\\&&
Insertion Orbit에서 어느정도 거리를 두면서도,
\\&&
 에너지가 적게 드는 낮은 궤도를 선택해야 한다.
\\150\,km$\pm$5\,km&&
\\200\,km$\pm$5\,km&재급유 궤도& 광물 채취 왕복선이나 원정을 가는 우주선이 재급유
\\&&
 할 때 들를 수 있는 궤도. 바깥쪽에서 접근하기 
 \\&&
쉽도록 높은 고도의 궤도로 정한다.
\\2863.334\,06\,km&정지 궤도\textsuperscript{!}&
\\\hline
\end{tabular}
\begin{tablenotes}
\item[!] \url{http://wiki.kerbalspaceprogram.com/wiki/KEO}
\item[주의사항:] 궤도간 이동을 할 때는 충돌을 피하기 위해 다른 물체의 궤도면을 피해서 이동할 것. 특히 적도 궤도면을 피해 5도 정도의 각도를 주고 이동할 것.
\end{tablenotes}
\end{threeparttable}
\end{adjustbox}
\begin{table}[h]
\caption{원형궤도 및 타원궤도에서의 원지점/근지점에서의 속도(m/s). 첫행의 고도에서의 속도가 표에 적혀있는 내용. 각 행의 첫 열은 궤도 반대편의 고도이다. 한쪽이 원지점이면 다른쪽이 근지점이 된다.}
\begin{tabular}{|r|r|r|r|r|r|r|}
\hline
&80\,km&100\,km&120\,km&150\,km&200\,km&2863\,km
\\\hline
80\,km&2278.93&2229.80&2182.86&2116.20&2014.09&578.54%540
\\\hline
100\,km&2295.39&2246.14&2199.07&2132.24&2029.83&585.57%574
\\\hline
120\,km&2311.26&2261.90&2214.72&2147.72&2045.03&592.46%459
\\\hline
150\,km&2334.04&2284.54&2237.21&2169.98&2066.91&602.52%520
\\\hline
200\,km&2369.52&2319.80&2272.26&2204.70&2101.07&618.62%620
\\\hline
2863\,km&2946.58&2897.20&2849.84&2782.30&2678.11&1009.81
\\\hline
\end{tabular}
\end{table}
\section{달 착륙}
실제 아폴로 착륙시에는 3단계의 매뉴버를 하였다고 한다. 우선 착륙선을 감속하여 목표지점보다 가까이 떨어지도록 한다(수평속도 감속). 그 다음, 떨어지는 속도를 감속하여 목포지점보다 약간 멀리 떨어지도록 감속한다(기존에 떨어지는 방향보다 수직방향의 속도를 우선 감속). 마지막으로 미세조정 및 역분사로 착륙의 마지막 단계이다. 이를 참고하여 제안해보고자 한다.


\chapter{대기모델}%%이건 어디 넣죠
\section{Kerbin}
\subsection{Pressure}
Kerbin의 Pressure는 표준모델을 따름.
\subsection{Temperature}
Temperature는 햇빛영향을 받음. 측정필요