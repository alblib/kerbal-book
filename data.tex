\part{데이터 및 프로토콜}
\chapter{게임 데이터}
이 챕터의 내용은 주로 게임 내부 데이터 및 물리법칙에 대한 것이며, 
따라서 저작권은 게임제작사에 있으며, 인용하고 있는 제3자의 저작권은 없는 것으로 해석합니다. (독창성 결여)
%데이터는 게임 플레이 중 직접 확인할 수 있는 부분이며, 상이한 점을 발견하면 갱신 부탁드립니다.
데이터는 게임 플레이 중 직접 확인할 수 있는 부분이며, 상이한 점이 있다면 github에서 직접 갱신할 수 있다.

\section{행성 데이터}
KSP에서 행성과 위성의 운동은 물리법칙의 시뮬레이션에 의한 것이 아니라 고정된 주기를 가진 고정된 운동궤도에 따른 것입니다. 공전주기와 공전궤도는 항성 또는 모행성이 아주 무겁다고 생각하고 만들어진 1체문제의 해이며, 모행성의 위성에 의한 흔들림 등은 전혀 고려되어 있지 않습니다.
\begin{tabular}{|l|r|r|r|r|}
\hline
이름&질량$\times$중력상수&원일점&근일점&공전주기\footnote{원일점과 근일점의 종속변수이나, 컴퓨터의 계산오차가 있으므로 게임 내부적으로 상수일 것으로 생각됩니다.}
\end{tabular}
(*괄호 안의 물리량은 다른 양으로부터 유도되는 양입니다.)
(표에 주석은 어떻게 달죠)
\paragraph{1태양일}
두가지 요인: 공전궤도의 이심률과 자전축과 공전평면과의 각도.

\section{위성 데이터}
\section{엔진 데이터}

\chapter{}