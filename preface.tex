\chapter*{Kerbal Space Program이란?}
Kerbal Space Program (약칭 KSP) 란 Squad사에서 만든 우주선 시뮬레이션 게임입니다.
간단히 말하자면, 이 게임은 뉴턴역학을 충실히 구현해 로켓역학을 배울 수 있는, 체험할 수 있는 게임입니다.
다만, 좀 더 하드코어 시뮬레이션이지만 (실제 아폴로 미션을 구현하는 등) 한정된 상황만을 체험해 볼 수 있는
다른 우주선 게임과는 다르게,
샌드박스적 요소가 강한 게임이라고 할 수 있겠습니다.

%여러가지 물리현상 (스킬) 소개
이 게임에서는
뉴턴 제2법칙 등과 같은 뉴턴 역학을 단순히 교과서에서 배우는 것만으로는 알기 힘든
여러가지 물리적 현상들을 직접적 피드백을 통해 손수 익힐 수 있습니다.
게임 맵 상의 궤도 위에서 
로켓분사(매뉴버) 계획을 하면, 얼마만큼 궤도가 변하는지 실시간으로 보여주므로,
뉴턴역학을 전혀 모른다고 하더라도 직접 만져보면서 흥미로운 로켓공학의 현상들을 익힐 수 있고 
미션을 성공시켜 볼 수 있습니다.
(처음 궤도에 올리는 데에는 이런 느긋한 방법이 통하지 않을지도 모릅니다. )


이 게임은 그저 실제에 충실히 뉴턴역학을 구현하는 데에만 신경을 쓴 것은 아닙니다. 
실제 태양계보다 축소된 모델로 우주선을 궤도에 올리는 것을 실제보다 용이하게 해놓은 것이 많은 사용자들이 더 쉽게 플레이 해볼 수 있게 하는 장점을 가지고 있습니다.
게임에 등장하는 우주비행사도 인간이 아닌 작고 귀여운 녹색인간으로, 캐릭터성 또한 겸비하고 있습니다.
이와 같은 쉬운 게임 전개는 다음과 같은 코멘트가 많은 추천을 받게 하기도 했습니다:
\emph{딸도 몇 번의 시행착오를 거쳐서 달까지 우주선을 보낼 수 있었다.}


이처럼 학생들에게 물리를 배울 동기를 제공하거나 로켓공학의 기초지식을 전달하는 데에 효과가 있고,
제작사에서도 교육용을 따로 판매하고 있을 정도로 교육용도로서의 홍보에 적극적입니다.
학교에서 단체로 구매한 다음, 학생들에게 숙제를 내어줄 수도 있습니다.
또한, 2차 창작에도 유연하여, 링크만 건다면 게임 영상을 유튜브에 올릴 수 있도록 홈페이지에서 명시적으로 허락하고 있습니다.
많은 이용자들이 자신만의 우주선이나 비행 방법을 올리고 있고, 어떤 사람은 멋진 우주 영화를 만든 사람도 있습니다.
이처럼 다른 게임에 비해 오픈된 마인드와 샌드박스이면서도 하드코어 하지 않은 게임성 덕분에
더욱더 많은 이용자들이 유입되고 있고 앞으로의 기능확장에도 무궁무진한 가능성이 열려 있는 게임이라고 할 수 있습니다.


하지만, 새롭게 이 게임을 시작하는 사람들이 진행에 많은 어려움을 겪는 것을 보았습니다. 
처음 궤도에 우주선을 올릴 때에는 단순한 관성계의 뉴턴역학만으로는 성공시킬 수 없기 때문입니다.
첫 발사를 성공시키기 위해서는 공기역학을 이해해야 하고, 게임상의 엔진의 성질에 대해서도 이해할 필요가 있습니다.

이 서적의 목적은 크게 두 가지입니다. 
우선, 처음 입문하는 사람들이 쉽게 따라하면서 게임을 이해할 수 있는 입문서로서의 목적이 있습니다.
또한, 가능한한 많은 수치표를 넣어 숙련자들도 이용할 수 있는 계산툴로서 기능하게 하는 목적이 있습니다.
최대한 첫째 목적을 이루는 것이 주 목표이며, 이를 지키면서도 최대한 많은 물리현상들을 설명하고, 또 정교한 스킬들을 익힐수 있게 하고자 합니다.
짧은 집필기간과 계속되는 게임의 업데이트 때문에 초판은 충분히 잘 기술되지 못할 수 있을 것입니다.
게임소스를 제외한 본 서적의 핵심적인 부분인 기술적인 설명과 저자의 개인연구 부분은 오픈소스 라이센스 하에 있으며, 앞으로 독자들께서도 자유롭게 편집할 수 있도록 할 것이오니,
혹시 틀린 점을 발견하거나 추가하고 싶은 내용이 있으면 책 마지막 장의 소스 URL에서 추가해주시길 부탁드립니다.

