\chapter*{Kerbal Space Program이란?}
Kerbal Space Program (약칭 KSP) 란 Squad사에서 만든 우주선 시뮬레이션 게임입니다.
뉴턴역학, 로켓역학

이 게임은 그저 실제에 충실히 뉴턴역학을 구현하는 데에만 신경을 쓴 것은 아닙니다. 
실제 태양계보다 축소된 모델로 우주선을 궤도에 올리는 것을 실제보다 용이하게 해놓은 것이 많은 사용자들이 더 쉽게 플레이 해볼 수 있게 하는 장점을 가지고 있습니다.
게임에 등장하는 우주비행사도 인간이 아닌 작고 귀여운 녹색인간으로, 캐릭터성 또한 겸비하고 있습니다.
이와 같은 쉬운 게임 전개는 다음과 같은 코멘트가 많은 추천을 받게 하기도 했습니다:
\emph{딸도 몇 번의 시행착오를 거쳐서 달까지 우주선을 보낼 수 있었다.}


여러가지 물리현상 (스킬) 소개

뉴턴 제2법칙 등과 같은 뉴턴 역학을 단순히 교과서에서 배우는 것만으로는 알 수 없는
여러가지 물리적 현상들을 직접적 피드백을 통해 손수 익힐 수 있습니다.

로켓분사(매뉴버)를 하였을 때 얼마만큼 궤도가 변하는지 실시간으로 보여주므로 

이처럼 교육 효과가 있고
제작사에서도 교육용을 따로 판매하고 있을 정도로 교육용도로서의 홍보에 적극적입니다.
학교에서 단체로 구매한 다음, 학생 개개인에게 '달에 갔다오라'고 숙제를 내어줄 수 있을 정도입니다.
또한, 2차 창작에도 유연하여, 링크만 건다면 게임 영상을 유튜브에 올릴 수 있도록 홈페이지에서 명시적으로 허락하고 있습니다.
무궁무진한 가능성



하지만, 새롭게 이 게임을 시작하는 사람들이 진행에 많은 어려움을 겪는 것을 보았습니다. 이 서적의 목적은 크게 두 가지입니다. 
우선, 처음 입문하는 사람들이 쉽게 따라하면서 게임을 이해할 수 있는 입문서로서의 목적이 있습니다.
또한, 가능한한 많은 수치표를 넣어 숙련자들도 이용할 수 있는 계산툴로서 기능하게 하는 목적이 있습니다.
최대한 첫째 목적을 이루는 것이 주 목표이며, 이를 지키면서도 최대한 많은 물리현상들을 설명하고, 또 정교한 스킬들을 익힐수 있게 하고자 합니다.
짧은 집필기간과 계속되는 게임의 업데이트 때문에 초판은 충분히 잘 기술되지 못할 수 있을 것입니다.
게임소스를 제외한 본 서적의 핵심적인 부분인 기술적인 설명과 저자의 개인연구 부분은 오픈소스 라이센스 하에 있으며, 앞으로 독자들께서도 자유롭게 편집할 수 있도록 할 것이오니,
혹시 틀린 점을 발견하거나 추가하고 싶은 내용이 있으면 책 마지막 장의 소스 URL에서 추가해주시길 부탁드립니다.

